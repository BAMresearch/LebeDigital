

The general idea of this analytical homogenization procedure is to describe the overall stiffness of a body $\body$, based on the properties of the individual phases, i.e. the matrix and the inclusions.
Each of the $n$ phases is denoted by the index $\phaseIndex$, where $\phaseIndex = 0$ is defined as the matrix phase.
The volume fraction of each phase is defined as
\begin{align}
	\volFracPhase = \frac{\left\| \bodyPhase \right\|}{\left\| \body \right\|} \quad  \text{for}~ \phaseIndex = 0, ..., n.
\end{align}
The inclusions are assumed to be spheres, defined by their radius $\radiusPhase$.
The shells are defined by their outer radius, their thickness follows as the difference to the inner inclusions.
The elastic properties of each homogeneous and isotropic phase is given by the material stiffness matrix $\bL^{(\phaseIndex)}$, here written in terms of the bulk and shear moduli $\bulkMod$ and $\shearMod$,
\begin{align}
	\matStiffPhase= 3 \bulkModPhase \orthProjV + 2 \shearModPhase \orthProjD  \quad \text{for}~ \phaseIndex = 0, ..., n, \label{eq:Lr}
\end{align}
where $\orthProjV$ and $\orthProjD$ are the orthogonal projections of the volumetric and deviatoric components.\\
The methods assumes that the micro-heterogeneous body $\body$ is subjected to a macroscale strain $\strain$.
It is assumed that for each phase concentration factor $\concentrationPhase$ can be defined such that
\begin{align}
	\strainPhase = \concentrationPhase\strain \quad  \text{for}~ \phaseIndex = 0, ..., n, \label{eq:strainaverage}
\end{align}
which computes the average strain $\strainPhase$ based on the overall strains.
This can then be used to compute the effective stiffness matrix $\matStiffEff$ as a volumetric sum over the constituents weighted by the concentration factor 
\begin{align}
	\matStiffEff = \sum_{\phaseIndex=0}^{n} \volFracPhase \matStiffPhase\concentrationPhase \quad  \text{for}~ \phaseIndex = 0, ..., n.\label{eq:Leff}
\end{align}

The concentration factors $\concentrationPhase$,
\begin{align}
	\concentrationZero &= \left( \volFracZero\bI + \sum^{n}_{\phaseIndex=1} \volFracPhase \dilConcentrationPhase\right)^{-1}\label{eq:A0}\\
	\concentrationPhase &= \dilConcentrationPhase\concentrationZero\quad  \text{for}~ \phaseIndex = 1, ..., n,
\end{align}
are based on the dilute concentration factors $\dilConcentrationPhase$, which need to be obtained first.
The dilute concentration factors are based on the assumption that each inclusion is subjected to the average strain in the matrix $\strainZero$.
\begin{align}
	\strainPhase = \dilConcentrationPhase\strainZero \quad  \text{for}~ \phaseIndex = 1, ..., n. 
\end{align}


The dilute concentration factors neglect the interaction among phases and are only defined for the inclusion phases $\phaseIndex = 1,...,n$.
The applied formulation uses an additive volumetric-deviatoric split. where
\begin{align}
	\dilConcentrationPhase = \dilConcentrationVPhase\Ivol +  \dilConcentrationDPhase \Idev \quad  \text{for}~ \phaseIndex = 1, ..., n,
\end{align}
This chosen method extends the basic Mori-Tanaka method the coated inclusions, following \cite{her_1993_nlib}, therefore two different formulations for the dilute concentration factors are given for the uncoated and coated inclusion. 

\paragraph{Dilute concentration factors for uncoated inclusions}
The formulations of the dilute concentration factors for an uncoated inclusion $\phaseIndex$ are
\begin{align}
	\dilConcentrationVPhase = \dfrac{\bulkModZero}{\bulkModZero + \auxAlphaZero(\bulkModPhase - \bulkModZero)}, \\
	\dilConcentrationDPhase = \dfrac{\shearModZero}{\shearModZero + \auxBetaZero(\shearModPhase - \shearModZero)}, 
\end{align}
with the auxiliary factors following from the Eshelby solution as
\begin{align}
	\auxAlphaZero = \frac{1 + \poissionZero}{3(1+ \poissionZero)} \quad\text{and}\quad 
	\auxBetaZero = \frac{2(4 - 5\poissionZero)}{15(1 - \poissionZero)}
\end{align}
where  $\poissionZero$ refers to the Poission's ratio of the matrix phase.
\paragraph{Effective stiffness}
Now that the formulation for the dilute concentration factor for inclusions are defined the effective bulk and shear modului can be computed based on a sum over the phases
\begin{align}
\bulkModEff = \dfrac{\volFracZero\bulkModZero + \sum^{n}_{\phaseIndex=1} \volFracPhase \bulkModPhase \dilConcentrationVPhase}{\volFracZero + \sum^{n}_{\phaseIndex=1} \volFracPhase \dilConcentrationVPhase},\label{eq:keff} \\
\shearModEff = \dfrac{\volFracZero\shearModZero + \sum^{n}_{\phaseIndex=1} \volFracPhase \shearModPhase \dilConcentrationDPhase}{\volFracZero + \sum^{n}_{\phaseIndex=1} \volFracPhase \dilConcentrationDPhase}.\label{eq:geff}
\end{align}
\paragraph{Average stress in matrix}

Based on the concept of \eqref{eq:strainaverage}, with the formulations \eqref{eq:Lr},\eqref{eq:Leff} and \eqref{eq:A0}, the average matrix stress is defined as 
\begin{align}
\stressZero = \matStiffZero\concentrationZero {\matStiffEff}^{-1}\stress. \label{eq:matrixstress}
\end{align}


