\subsection{Finite Element Concrete Model}
The notable feature of the concrete model is the evolution of the mechanical properties over time.
When concrete is mixed its consistency is close to liquid.
Due to a chemical reaction of the binder with water, called hydration, crystals develop that give concrete its strength and stiffness.
The reaction is both exothermal and temperature dependent.
Therefore, the primary model computes the temperature field $\temp$ \eqref{eq:heat1} and the degree of hydration $\DOH$ \eqref{eq:doh}.
The temperature model depends on three material properties, the effective thermal conductivity $\thermCondEff$, the specific heat capacity $C$ and the heat release $\dQdt$.
The $\dQdt$ in turn is governed by the hydration model, characterized by six parameters:
$\hydParBone, \hydParBtwo, \hydParEta, \tempRef, \activE$ and $\DOHmax$.
The first three $\hydParBone, \hydParBtwo$ and $\hydParEta$ are phenomenological, numerical parameters characterizing the shape of the function of the heat release.
$\tempRef$ does not need to be identified.
It is the reference temperature for which the first three parameters are calibrated.
$\activE$ is the activation energy, defining how the model will react to temperature changes, relative the the reference. 
$\DOHmax$ is the maximum degree of hydration that can be reached.
Following \cite{Mills1966fico}, the maximum degree of hydration is estimated, based on the water to binder ratio $\wc$, as $\DOHmax = \frac{1.031\,\wc}{0.194 + \wc}$.
As the degree of hydration if difficult to quantify experimentally, the heat release is used as a proxy.
By assuming the DOH is the fraction of the currently released heat with respect to its theoretical potential $\heatInf$, the current degree of hydration is estimated as $\DOH(\zeit) = \frac{\heat(\zeit)}{\heatInf}$.
As the potential heat release is also difficult to measure, as it takes a long time to fully hydrate and will only do so under perfect conditions, we obtain this value along side the other using the parameter identification method.
Figure \ref{fig:heatrelease} shows the influence of the three numerical parameter and the potential heat release on the heat release rate as well as the cumulative heat release.
\begin{figure}[h]%
	\centering
	\includegraphics[width=1.0\textwidth]{../figures/\heatReleasePlot}
	\caption{Influence of the hydration parameters on the heat release rate and the cumulative heat release.}\label{fig:heatrelease}
\end{figure}
For a detailed model description see in Appendix \ref{appendix:fem}.
In addition to influencing the reaction speed, the computed temperature is used to verify that the maximum temperature during hydration does not exceed a limit of $\tempLimit = \inputtemperaturelimit$\textdegree C.
Above this temperature, certain crystals start to revert back to a different state, expanding in volume and leading to cracks in the concrete.
This is implemented as a constraint for the optimization problem \eqref{eq:concstraintT}.
Based on the degree of hydration, the Young's modulus $\eMod$ of a linear-elastic material model
is approximated \eqref{eq:EwrtDOH}.
Further, a compressive strength in terms of the degree of hydration is computed \eqref{eq:fcwrtDOH}, which is utilized to determine a failure criterion based on the computed local stresses \eqref{eq:constraintStress}.
For a detailed description of the parameters evolution with respect to the degree of hydration see in Appendix \ref{appendix:fem_evolution}.
Figure \ref{fig:parameterEvolution} shows the influence of the different parameters.
In addition to the presented formulations in \cite{car_2016_mamt} which depend a theoretical value of parameters for fully hydrated concrete at $\DOH = 1$, this work reformulates the equations, to depend on the 28 day values $\eModTwentyEight$ and $\fcTwentyEight$ as well as the corresponding $\DOHTwentyEight$ which is obtained via a simulation.
\begin{figure}[h]%
	\centering
	\includegraphics[width=1.0\textwidth]{../figures/\evolutionPlot}
	\caption{Influence of parameters $\DOHt, \stiffExp$ and $\strengthCExp$ on evolution the Young's modulus and the compressive strength with respect to the degree of hydration $\DOH$. }\label{fig:parameterEvolution}
\end{figure}