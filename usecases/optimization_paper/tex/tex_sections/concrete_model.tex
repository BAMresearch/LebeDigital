\subsection{Finite Element Concrete Model}
The main feature of the concrete model is the evolution of the mechanical properties over time.
When concrete is mixed it is close to liquid.
Due to a  chemical reaction of the binder with water, called hydration, crystals develop that give concrete its strength and stiffness.
The reaction is both exothermal and temperature dependent.
Therefore, the primary model computes the temperature field $\temp$ \eqref{eq:heat1} and the degree of hydration $\DOH$ \eqref{eq:doh}.
The temperature model depends on three material properties, the effective thermal conductivity $\thermCondEff$, the specific heat capacity $C$ and the heat release $\dQdt$.
The $\dQdt$ in turn is governed by the hydration model, characterized by six parameters:
$\hydParBone, \hydParBtwo, \hydParEta, \tempRef, \activE$ and $\DOHmax$.
The first three $\hydParBone, \hydParBtwo$ and $\hydParEta$ are phenomenological, numerical parameters describing the function of the heat release.
$\tempRef$ does not need to be identified.
It is the reference temperature for which the first three parameters are calibrated.
$\activE$ is the activation energy, defining how the model will react to temperature changes, relative the the reference. 
$\DOHmax$ is the maximum degree of hydration that can be reached.
We follow REFERENCE, and compute this value based on WATER BINDER RATIO:::


...
FIGURE SHOWING Q over DOH???
CONTINUE HERE!!!!

The release heat is computed by the hydration model, ... characterized by 4 parameters....

For a detailed model description see in Appendix \ref{appendix:fem}.
In addition to influencing the reaction speed, the computed temperature is used to verify that the maximum temperature during hydration does not exceed a limit of $\tempLimit = \inputtemperaturelimit$\textdegree C.
Above this temperature, certain crystals start to revert back to a different state, expanding in volume and leading to cracks in the concrete.
This is implemented as a constraint for the optimization problem \eqref{eq:concstraintT}.
Based on the degree of hydration, the Young's modulus $\eMod$ of a linear-elastic material model
is approximated \eqref{eq:EwrtDOH}.
Further, a compressive strength in terms of the degree of hydration is computed \eqref{eq:fcwrtDOH}, which is utilized to determine a failure criterion based on the computed local stresses \eqref{eq:constraintStress}.
In the "origianl" model, REFERENCE, the models input requires a Young's modulus and compressive strength for the fully hydrated concrete, $\DOH = 1$.
As the maximum degree of hydration is assumed to depend on the water cement ratio $\wc$, RFERENCE to equation!!!, and .. it will.
..  


....
Figure xy shows the time evaolution of DOH, E and fc for the set of paramters given in Table XY, with some variations relevant to the respective models, as presented in the figure..



