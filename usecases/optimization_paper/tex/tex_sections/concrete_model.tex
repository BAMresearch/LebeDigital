\subsection{Finite Element Concrete Model}
The focus of the concrete model is on the evolution of the mechanical properties over time.
When concrete is mixed it is basically liquid.
Due to a  chemical reaction of the binder with water, called hydration, crystals develop that give concrete its strength and stiffness.
The reaction is both exothermal and temperature dependent.
Therefore, the primary model computes the temperature field $\temp$ and the degree of hydration $\DOH$, details given in Appendix \ref{appendix:fem}.
Based on the degree of hydration, the Young's modulus $\eMod$ of a linear-elastic material model
is approximated \eqref{eq:EwrtDOH}.
In addition, a compressive strength in terms of the degree of hydration is computed \eqref{eq:fcwrtDOH}, which is then used to determine, a failure criterion based on the computed local stress.
The simulation is used to compute two practical constraints relevant to the precast concrete industry.
At each time step, the worse point is chosen to represent the part, therefore ensuring that the criterion is fulfilled in the whole domain.
The first constraint limits the maximum allowed temperature to $\tempLimit = \inputtemperaturelimit$\textdegree C.
Above this temperature, certain crystals start to revert back to a different state, expanding in volume and leading to cracks in the concrete.
The constraint is computed as the normalized difference between the maximum temperature reached $\tempMax$ and the temperature limit $\tempLimit$ 
\begin{align}
\FEMConstraintT = \frac{\tempMax - \tempLimit}{\tempLimit},
\end{align}
where $\FEMConstraintT > 0$ is not admissible, as the temperature limit has been exceeded.
The second constraint is the estimated time of demolding.
This is critical, as the manufacturer has a limited number of forms.
The faster the part can be demolded, the faster is can be reused, increasing the output capacity.
On the other hand, the part must not be demolded too early, as it might get damaged while being moved.
To approximate the minimal time of demolding, a constraint is formulated based on the local stresses $\FEMConstraintStress$.
It evaluates the Rankine criterion for the principal tensile stresses, using the yield strength of steel $\beamfs$ and a simplified Drucker-Prager criterion, based on the evolving compressive strength of the concrete $\fc$,
\begin{align}
	\FEMConstraintStress =  \max
	\begin{cases}
		\FEMConstraintRK = \frac{\|\principalStressTension\| - \beamfs}{\beamfs} \\
		\FEMConstraintDP =  \frac{\sqrt{\frac{1}{3} \firstStressInvariant^2 - \secondStressInvariant} - \frac{\fc^3}{\sqrt{3}}}{\fc}
	\end{cases},
\end{align}
where $\FEMConstraintStress > 0$ is not admissible.
In contrast to standard yield surfaces, the value is normalized, to be unit less.
This constraint aims to approximate the compressive failure often simulated with plasticity and the tensile effect of reinforcement steel.
As boundary conditions a simply supported beam under it own weight has been chosen, to approximate possible loading condition while the part is moved.