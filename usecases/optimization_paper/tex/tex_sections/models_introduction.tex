The workflow (Figure \ref{fig:snakemake_workflow}) uses three different types of models. 
\ET{Maybe we need to mention the trained neural nets within the workflow here as well?}
The first is the homogenization approach, which computes effective concrete properties based on cement paste and aggregate data.
This model is an analytical function.
The second is a finite element model with two complex constitutive models incorporated.
A hydration model, which computes the evolution of the degree of hydration, considering the local temperature and the heat released during the reaction.
The mechanical model assumes isotropic, linear elastic material properties, that depend on the degree of hydration, therefore approximating the time evolution of the mechanical properties.
The third model is based on the procedure of a norm, to estimate the number of reinforcements within a beam.
Subsequent sections will provide insights into how these models function within the optimization framework.
\begin{figure}[ht]%
	\centering
	\includegraphics[width=1.0\textwidth]{../figures/\snakemakeGraph}
	\caption{snakemake workflow \ET{This pic should/could be included in the paper somewhere. I assume this need to be moved to the place where the actual optimization workflow is described.}}\label{fig:snakemake_workflow}
\end{figure}