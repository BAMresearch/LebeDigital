\subsection{Beam design}
Following design code \citeauthor{DIN1992-1-1} for a singly reinforced beam, meaning a reinforced concrete beam with only reinforcement at the bottom.
The assumed cross section is rectangular

\subsubsection{Maximum bending moment}
Assuming a simply supported beam with a given length $\beamLength$ in mm, a distributed load $\beamDistrLoad$ in N/mm and a point load $\beamPointLoad$ in N/mm
the maximum bending moment $\beamMaxMoment$ in N/mm$^2$ is computed as
% see Schneider (20. Auflage) 4.2
\begin{align}
	\beamMaxMoment= \beamDistrLoad \frac{\beamLength^2}{8} + \beamPointLoad \frac{\beamLength}{4}.
\end{align}
The applied loads already incorporate any required safety factors.
\subsubsection{Computing the minimal required steel reinforcement}
Given a beam with the height $\beamHeight$ in mm, a concrete cover of $\beamCover$ in mm, a steel reinforcement diameter of $\beamSteelDiameter$ in mm for the longitudinal bars and a bar diameter of $\beamSteelDiameter$ in mm for the transversal reinforcement also called stirrups,
\begin{align}
	\beamHeightEff = \beamHeight - \beamCover - \beamSteelDiameterStirrups - \frac{1}{2} \beamSteelDiameter.
\end{align}
According to the German norm standard safety factors are applied, $\beamTimeSF = 0.85$, $\beamConcreteSF = 1.5$ and $\beamSteelSF = 1.15$, leading to the design compressive strength for concrete $\beamfcd$ and  the design tensile yield strength $\beamfsd$ for steel
\begin{align}
	\beamfcd = \beamTimeSF \frac{\fc }{\beamConcreteSF}\\
	\beamfsd = \frac{\beamfs}{\beamSteelSF},
\end{align}
where $\fc$ denotes the concrete compressive strength and $\beamfs$ the steel's tensile yield strength.\\
% following ravis implementation
% bending moment with stress block - K
To compute the force applied in the compression zone, the lever arm of the applied moment $\beamLeverMoment$ is given by 
\begin{align}
	\beamLeverMoment &= \beamHeightEff(0.5+\sqrt{0.25-0.5 \beamK}),\quad \text{with}\quad
	\beamK = \frac{\beamMaxMoment }{\beamWidth \beamHeightEff^2 \beamfcd}. \label{eq:compr_str_constr}
\end{align}
The minimum required steel $\beamSteelReq$ is then computed based on the lever arm, the design yield strength of steel and the maximum bending moment, as
\begin{align}
	\beamSteelReq = \frac{\beamMaxMoment}{\beamfsd \beamLeverMoment}.
\end{align}
\subsubsection{Compressive strength constraint}
Based on Eq. \ref{eq:compr_str_constr}, we define the compressive strength constraint as
\begin{align}
	\beamConstraintFc = \beamK - 0.5, 
\end{align}
where $\beamConstraintFc > 0$ is not admissible, as there is no solution for \ref{eq:compr_str_constr}.
\subsubsection{Geometrical constraint}
For our simplified example, we assume the steel reinforcement is only arrange in one layer.
According to \citeauthor{DIN1992-1-1}, the minimum spacing between two bars $\beamMinSpacing$ is given by
\begin{equation}
	\beamMinSpacing = \max
	\begin{cases}
		\inputconcretecover \,\text{\inputconcretecoverunit}\\
		$\beamSteelDiameter$
	\end{cases}       
\end{equation}
Effective width
\begin{equation}
	\beamWidthEff = \beamWidth - 2 \beamMinSpacing - 2 \beamSteelDiameterStirrups
\end{equation}
Find max number for steel bars $\beamNSteelMax$ and maximal diameter $\beamSteelDiameterMax$ for a given list of admissible diameters that fulfill
\begin{equation}
	s \ge \beamMinSpacing,\quad\text{with}\quad s = \frac{\beamWidthEff - \beamNSteelMax \beamSteelDiameterMax}{\beamNSteelMax - 1} \quad\text{and}\quad \beamNSteelMax \ge 2
\end{equation}
The maximum possible reinforcement is given by
\begin{equation}
	\beamSteelMax = \beamNSteel \pi \left( \frac{\beamSteelDiameter}{2}\right)^{2}
\end{equation}
The geometry constraint is computed as
\begin{equation}
	\beamConstraintGeometry = \frac{\beamSteelReq - \beamSteelMax}{\beamSteelMax}
\end{equation}
where $\beamConstraintGeometry > 0$ is not admissible, as the required steel area exceeds the available space.
To simplify the optimization procedure, the two constraints are combined into one

\begin{equation}
	\beamConstraintBeam = \beamConstraintGeometry * \beamConstraintFc
\end{equation}




-....%\beamSteelReq


....

The maximum ....


required area  - max steel area / max area (reinforcement)


FINISH THIS!!!!!



