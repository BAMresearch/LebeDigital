\subsection{Homogenized Concrete Parameters}
Experimental data generally provided for concrete material research includes measurements both on concrete and on cement specimens.
For the data regarded within this work, the measurements of compressive strength are performed on concrete, the calorimetry on cement samples.
Properties of the aggregates are available regardless of the cement.\\
\begin{figure}[b]%
	\centering
	\includegraphics[width=1.0\textwidth]{../figures/\homogenizationPlot}
	\caption{Influence of aggregate ratio on effective concrete properties}\label{fig:homogenization}
\end{figure}
To be able to take concrete as well as cement data into account, an analytical homogenization procedure is used.
The homogenized effective concrete properties are the Young's modulus $\eMod$, the Poisson's ratio $\poission$, the compressive strength $\fc$, the density $\density$, the thermal conductivity $\thermCond$, the heat capacity $\heatCapSpecific$ and the total heat release $\heatInf$.
Depending on the physical meaning, these properties need slightly different methods to estimate the effective concrete properties.
To visualize the qualitative effect of the amount of aggregates within the concrete, effective properties are plotted for aggregate values from 0 to 1, respectively pure cement to only aggregates, see Figure \ref{fig:homogenization}.
Note that both extremes are purely numerical examples.
\mbox{Table \ref{tab:homogenizationproperties}} gives an overview of the material properties of the constituents used in the subsequent simple examples.
For the considered example properties, the relations are close to linear, this can change, when the difference between the matrix and the inclusion properties is increased or more complex micro mechanical mechanisms are incorporated, as air pores or the interfacial transition zone.
These can be considered within the chosen homogenization scheme by adding further phases, c.f. \cite{nee_2012_ammf}.
The required data for a meaningful calibration is however out of the scope of the available data.
\begin{table}[ht]
	\begin{center}
		\begin{minipage}{.9\textwidth}
			\caption{Properties of the micro scale phases used in subsequent examples}\label{tab:homogenizationproperties}
			\begin{tabular}{lccccccc}
				\toprule
				Phase & $\eMod$ & $\poission$ & $\fc$ & $\density$ & $\thermCond$ & $\heatCapSpecific$ & $\heatInf$\\
				  & $\pasteEunit$  & $-$  & $\pastefcunit$  & $\pasterhounit$  & $\pasteCunit$  & $\pastekappaunit$  &  $\pasteQunit$ \\
				\midrule
			Paste	& \pasteE & \pastenu & \pastefc & \pasterho & \pasteC & \pastekappa &  \pasteQ \\
			Aggregates	& \aggregatesE & \aggregatesnu & - & \aggregatesrho & \aggregatesC & \aggregateskappa &  0 \\
				\botrule
			\end{tabular}
		\end{minipage}
	\end{center}
		
\end{table}
\\
The elastic, isotropic properties $\eMod$ and $\poission$ of the concrete are approximated using the Mori-Tanaka homogenization scheme \cite{mor_1973_asi}.
The method assumes spherical inclusions in an infinite matrix and considers the interactions of multiple inclusions. Details given in \ref{ssec:mt_elastic}\\
The estimation of the concrete compressive strength $\fcEff$ follows the ideas of \cite{nev_2018_mcam}.
The theory is that a failure in the cement paste will cause the concrete to crack.
The approach is based on two main assumptions.
First, the Mori-Tanaka method is used to estimate the average stress within the matrix material $\stressMatrix$. 
Second, the von Mises failure criterion of the average matrix stress is used to estimate the uniaxial compressive strength.
The effects shown in Figure \ref{fig:homogenization} depends on the difference between the Young's modulus of the paste and the aggregates. Details given in \ref{ssec:compressivestrength}.\\
Homogenization of the thermal conductivity is also generally based on the Mori-Tanaka method, following the ideas of \cite{str_2011_mbeo}. Details given in \ref{ssec:thermalconductivity}.\\
The remaining quantities can be directly computed based on their volume average.
This is the case for density $\density$, the heat capacity $\heatCapSpecific$ and the total heat release $\heatInf$.
As example for the volume averaged quantities, the heat release is shown in Figure \ref{fig:homogenization} as it exemplifies the expected linear relation of the volume average as well as the zero heat output of a theoretical pure aggregate.