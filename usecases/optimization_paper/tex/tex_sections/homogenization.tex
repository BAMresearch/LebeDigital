\subsection{Notes on micromechanic based concrete homogenization}

\cite{mor_1973_asi} original paper of mt method\\
\cite{str_2011_mbeo} mori tanaka homogenization of thermal conductivity\\
\cite{nev_2018_mcam} description of fc value\\
\cite{nee_2012_ammf} description of E homogenization\\
\cite{pic_2011_uqso} similar ansatz, multiscale homogenization including wc focus\\
\cite{her_1993_nlib} paper with correct coating formulation




This manuscript gives an overview of the implementation used within my concrete simulation.
The goal is to compute effective concrete parameters based on information of the micro constituents.
In particular the cement paste and aggregates are considered.
It is possible to consider air pores as well as the influence of the interfacial transition zone (IFZ) as a coat surrounding aggregates with reduced stiffness. 
First the estimation of the effective Young's modulus $\eModEff$ and Poisson's ratio $\poissionEff$ is given.
It is based on the classical analytical Mori-Tanaka homogenization formulation \cite{mor_1973_asi} for spherical inclusions in a matrix.
This is extended to the case of a coated inclusions, following \cite{her_1993_nlib}.
In general the implementation of the stiffness homogenization is following the formulations presented in \cite{nee_2012_ammf}, however the appendix includes some errors, where the original paper \cite{her_1993_nlib} should be consulted.
Subsequently the estimation of the effective compressive strength $\fcEff$ is given, following the ideas of \cite{nev_2018_mcam}.

\subsection{Stiffness homogenization as presented in \cite{nee_2012_ammf}}
The general idea of this analytical homogenization procedure is to describe the overall stiffness of a body $\body$, based on the properties of the individual phases, i.e. the matrix and the inclusions.
Each of the $n$ phases is denoted by the index $\phaseIndex$, where $\phaseIndex = 0$ is defined as the matrix phase.
The volume fraction of each phase is defined as
\begin{align}
	\volFracPhase = \frac{\left\| \bodyPhase \right\|}{\left\| \body \right\|} \quad  \text{for}~ \phaseIndex = 0, ..., n.
\end{align}
The inclusions are assumed to be spheres, defined by their radius $\radiusPhase$.
The shells are defined by their outer radius, their thickness follows as the difference to the inner inclusions.
The elastic properties of each homogeneous and isotropic phase is given by the material stiffness matrix $\bL^{(\phaseIndex)}$, here written in terms of the bulk and shear moduli $\bulkMod$ and $\shearMod$,
\begin{align}
	\matStiffPhase= 3 \bulkModPhase \orthProjV + 2 \shearModPhase \orthProjD  \quad \text{for}~ \phaseIndex = 0, ..., n, \label{eq:Lr}
\end{align}
where $\orthProjV$ and $\orthProjD$ are the orthogonal projections of the volumetric and deviatoric components.\\
The methods assumes that the micro-heterogeneous body $\body$ is subjected to a macroscale strain $\strain$.
It is assumed that for each phase concentration factor $\concentrationPhase$ can be defined such that
\begin{align}
	\strainPhase = \concentrationPhase\strain \quad  \text{for}~ \phaseIndex = 0, ..., n, \label{eq:strainaverage}
\end{align}
which computes the average strain $\strainPhase$ based on the overall strains.
This can then be used to compute the effective stiffness matrix $\matStiffEff$ as a volumetric sum over the constituents weighted by the concentration factor 
\begin{align}
	\matStiffEff = \sum_{\phaseIndex=0}^{n} \volFracPhase \matStiffPhase\concentrationPhase \quad  \text{for}~ \phaseIndex = 0, ..., n.\label{eq:Leff}
\end{align}

The concentration factors $\concentrationPhase$,
\begin{align}
	\concentrationZero &= \left( \volFracZero\bI + \sum^{n}_{\phaseIndex=1} \volFracPhase \dilConcentrationPhase\right)^{-1}\label{eq:A0}\\
	\concentrationPhase &= \dilConcentrationPhase\concentrationZero\quad  \text{for}~ \phaseIndex = 1, ..., n,
\end{align}
are based on the dilute concentration factors $\dilConcentrationPhase$, which need to be obtained first.
The dilute concentration factors are based on the assumption that each inclusion is subjected to the average strain in the matrix $\strainZero$.
\begin{align}
	\strainPhase = \dilConcentrationPhase\strainZero \quad  \text{for}~ \phaseIndex = 1, ..., n. 
\end{align}


The dilute concentration factors neglect the interaction among phases and are only defined for the inclusion phases $\phaseIndex = 1,...,n$.
The applied formulation uses an additive volumetric-deviatoric split. where
\begin{align}
	\dilConcentrationPhase = \dilConcentrationVPhase\Ivol +  \dilConcentrationDPhase \Idev \quad  \text{for}~ \phaseIndex = 1, ..., n,
\end{align}
This chosen method extends the basic Mori-Tanaka method the coated inclusions, following \cite{her_1993_nlib}, therefore two different formulations for the dilute concentration factors are given for the uncoated and coated inclusion. 

\subsubsection{Dilute concentration factors for uncoated inclusions}
The formulations of the dilute concentration factors for an uncoated inclusion $\phaseIndex$ are
\begin{align}
	\dilConcentrationVPhase = \dfrac{\bulkModZero}{\bulkModZero + \auxAlphaZero(\bulkModPhase - \bulkModZero)}, \\
	\dilConcentrationDPhase = \dfrac{\shearModZero}{\shearModZero + \auxBetaZero(\shearModPhase - \shearModZero)}, 
\end{align}
with the auxiliary factors following from the Eshelby solution as
\begin{align}
	\auxAlphaZero = \frac{1 + \poissionZero}{3(1+ \poissionZero)} \quad\text{and}\quad 
	\auxBetaZero = \frac{2(4 - 5\poissionZero)}{15(1 - \poissionZero)}
\end{align}
where  $\poissionZero$ refers to the Poission's ratio of the matrix phase.
\subsubsection{Effective stiffness}
Now that the formulation for the dilute concentration factor for inclusions are defined the effective bulk and shear modului can be computed based on a sum over the phases
\begin{align}
\bulkModEff = \dfrac{\volFracZero\bulkModZero + \sum^{n}_{\phaseIndex=1} \volFracPhase \bulkModPhase \dilConcentrationVPhase}{\volFracZero + \sum^{n}_{\phaseIndex=1} \volFracPhase \dilConcentrationVPhase},\label{eq:keff} \\
\shearModEff = \dfrac{\volFracZero\shearModZero + \sum^{n}_{\phaseIndex=1} \volFracPhase \shearModPhase \dilConcentrationDPhase}{\volFracZero + \sum^{n}_{\phaseIndex=1} \volFracPhase \dilConcentrationDPhase}.\label{eq:geff}
\end{align}
\subsection{Strength homogenization as presented in \cite{nev_2018_mcam}}
Based on the ideas of the previous section, a formulation is used to homogenize the effective strength of the composite.
The strength estimation is based on two main assumptions.
First assumptions, the Mori-Tanaka method is used to estimate the average stress within the matrix material, which is the indicator of overall failure.
Based on the concept of \eqref{eq:strainaverage}, with the formulations \eqref{eq:Lr},\eqref{eq:Leff} and \eqref{eq:A0}, the average matrix stress is defined as 
\begin{align}
	\stressZero = \matStiffZero\concentrationZero {\matStiffEff}^{-1}\stress. \label{eq:matrixstress}
\end{align}
Second assumption, the von Mises failure criterion is used to estimate the uniaxial compressive strength
\begin{align}
	\sqrt{3 \Jtwo} - {\fc} = 0. \label{eq:vonMises}
\end{align}
The second deviatoric stress invariant is defined as
\begin{align}
	\Jtwo = \frac{1}{2} \stressD:\stressD,\quad\text{with}\quad
	\stressD = \stress - \frac{1}{3}\tr(\stress)\bI.
\end{align}
The goal is to find a uniaxial macroscopic stress $\stress = \begin{bmatrix} -\fcEff & 0 & 0 &0&0&0 \end{bmatrix}\TP$ which exactly fulfills the von Mises failure criterion \eqref{eq:vonMises} for the average stress within the matrix $\stressZero$.
The procedure here is taken from the code provided in the link in \cite{nee_2012_ammf}.

First we compute the second deviatoric stress invariant $\JtwoTest$ for a uniaxial test stress $\stressTest = \begin{bmatrix} \forceTest & 0 & 0 &0&0&0 \end{bmatrix}\TP$. Then the matrix stress $\stressZero$ is computed based on the test stress following \eqref{eq:matrixstress}. This is used to compute the second deviatoric stress invariant $\JtwoZero$ for the average matrix stress.
Now the effective compressive strength is estimated as
\begin{align}
	\fcEff = \frac{\JtwoTest}{\JtwoZero} \forceTest.
\end{align}


\subsection{Thermal conductivity homogenization as presented in \cite{str_2011_mbeo}}
Homogenization the thermal conductivity is also based on the Mori-Tanaka method.
The formulations are similar as for the stiffness homogenization, i.e. \eqref{eq:keff} and \eqref{eq:geff}.
Here only the implemented formulation is given, no further background.
The expressions are taken from \cite{str_2011_mbeo}, Section 2.2.
The thermal conductivity $\thermCondHom$ is given as
\begin{align}
	\thermCondHom = \dfrac{\volFracZero\thermCond + \sum^{n}_{\phaseIndex=1} \volFracPhase \thermCondPhase \concentrationThermCondPhase}{\volFracZero + \sum^{n}_{\phaseIndex=1} \volFracPhase \concentrationThermCondPhase}\quad\text{and}\quad
	\concentrationThermCondPhase = \frac{3\thermCondZero}{2\thermCondZero+\thermCondPhase}.
\end{align}
