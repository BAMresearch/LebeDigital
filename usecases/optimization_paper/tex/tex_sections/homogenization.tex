\subsection{Homogenized Concrete Parameters}
Experimental data generally provided for concrete material research includes measurements both on concrete and on cement specimens.
For the data regarded within this work, the measurements of compressive strength are performed on concrete, the calorimetry on cement samples.
Properties of the aggregates are available regardless of the cement.\\

To be able to take concrete as well as cement data into account, an analytical homogenization procedure is used.
This allows for a continuous optimization of the aggregate amount, requiring few experimental data points.
The effective concrete properties that are homogenized are the Young's modulus $\eMod$, the Poission's ratio $\poission$, the compressive strength $\fc$, the density $\density$, the thermal conductivity $\thermCond$, the heat capacity $\heatCapSpecific$ and the total heat release $\heatInf$.
Depending on the physical meaning, these properties need different methods to estimate the effective concrete properties.
To visualize the effect of homogenization, effective properties are plotted for aggregate values from 0 to 1, respectively pure cement to only aggregates.
Note that both extremes are purely numerical results.
Due to the chosen homogenization method, values of aggregates above ??? should be considered nonphysical (????).
Table XXX gives an overview of the material properties of the constituents used in the subsequent simple examples.
\\

The elastic, isotropic properties $\eMod$ and $\poission$ of the concrete are approximated using the Mori-Tanaka homogenization scheme \cite{mor_1973_asi}.
The method assumes spherical inclusions in an infinite matrix and considers the interactions of multiple inclusions.
The effect of the aggregate amount on the effective property is shown in Figure XXX.
DESCRIBE THE FFECT

The estimation of the concrete compressive strength $\fcEff$ follows the ideas of \cite{nev_2018_mcam}.
The assumption is that a failure in the cement paste will cause the concrete to crack.
The approach is based on two main assumptions.
First, the Mori-Tanaka method is used to estimate the average stress within the matrix material $\stressMatrix$. 
Second, the von Mises failure criterion of the average matrix stress is used to estimate the uniaxial compressive strength.
As shown in Figure XXX, DECRIBE THE EFFECT!!!!

Homogenization of the thermal conductivity is also generally based on the Mori-Tanaka method, following the ideas of \cite{str_2011_mbeo}.
Values shown in Figure XXX, DECRIBE THE EFFECT!!!!

The remaining quantities can be directly computed based on their volume average.
This is the case for density $\density$, the heat capacity $\heatCapSpecific$ and the total heat release $\heatInf$.
The heat release is shown as example in Figure XXX as exemplifies the expected linear elatiion of the volume average.

