\subsubsection{Compute required reinforcement steel}
To ensure that the chosen mix and beam properties fully fulfill the requirements according to the standards, the computation of the required steel reinforcement is integrated into the workflow.
The defining characteristic is the applied moment.
It is mainly governed by the height of the beam and the external force.

... continue this once we have discussed how this new optimization goal con be incorporated.



\begin{table}[ht]
	\begin{center}
		\begin{minipage}{.9\textwidth}
			\caption{Input for the computation of the steel reinforcement}\label{tab:beamdesigninput}
			\begin{tabular}{lrl}
				\toprule
				Name &  Value&Unit\\
				\midrule
				Length & \beamExSpan &$\beamExSpanUnit$\\
				Width & \beamExWidth &$\beamExWidthUnit$\\
				Height& \beamExHeightC &$\beamExHeightCUnit$\\
				Steel yield strength& \beamExYieldStrSteel &$\beamExYieldStrSteelUnit$\\
				Diameter stirrups& \beamExSteelDiaBu &$\beamExSteelDiaBuUnit$\\
				Minimal concrete cover& \beamExCoverMin &$\beamExCoverMinUnit$\\
				Load& \beamExPointLoadC &$\beamExPointLoadCUnit$\\
				Concrete compressive strength& \beamExComprStrConcreteC &$\beamExComprStrConcreteCUnit$\\
				\botrule
			\end{tabular}
		\end{minipage}
	\end{center}
	
\end{table}






\begin{figure}[ht]%
	\centering
	\includegraphics[width=1.0\textwidth]{../figures/\beamDesignPlot}
	\caption{Influence of beam height, concrete compressive strength and load in the center of the beam on the required steel.}\label{fig:beamdesign}
\end{figure}


