\subsection{Beam design according to EC2}
To ensure that the chosen mix and beam properties fulfill the requirements according to the European norm for reinforced concrete, \cite{DIN1992-1-1}, the computation of the required steel reinforcement is integrated into the workflow.
For a given beam height, load and compressive strength, the minimum required cross section of the steel reinforcement is computed \eqref{eq:Areq}.
The results for example parameters, given in Table \ref{tab:beamdesigninput}, are visualized in Figure \ref{fig:beamdesign}.
To ensure that this design is realistic, the continuous cross section is transformed into a discrete number of bars, with a diameter chosen from a list.
This is visible in Figure \ref{fig:beamdesign} by the step-wise increase in cross sections.
The admissible results are restricted by two constrains.
One is coming from a minimal required compressive strength \eqref{eq:constraintfc}, visualized as dashed line.
The other, based on the available space to place bars with admissible spacing \eqref{eq:constraintGeo}, visualized as the dotted line.
Further detail on the computation are given in Appendix \ref{appendix:beam}.

\begin{figure}[ht]%
	\centering
	\includegraphics[width=1.0\textwidth]{../figures/\beamDesignPlot}
	\caption{Influence of beam height, concrete compressive strength and load in the center of the beam on the required steel. The dashed lines represent the minimum compressive strength constraint \eqref{eq:constraintfc}, the dotted lines the geometrical constraint from the spacing of the bars \eqref{eq:constraintGeo}.\label{fig:beamdesign}}
\end{figure}