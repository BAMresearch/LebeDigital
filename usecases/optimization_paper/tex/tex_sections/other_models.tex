\subsection{Consideration of reinforcement in FEM model}
The integration of discrete steel reinforcement within FE models is complex and beyond the scope of this publication's focus.
As an approximation, we account for its influence by considering the tensile strength within the stress constraint \ref{eq:constraintStress} to be equal to the yield stress of the steel reinforcement.
Moreover, the FEM simulation is simplifies by disregarding the impact of the reinforcement bars on the elastic modulus of the beam.\\

\ET{I am not sure if this needs such a prominent section, but it should be mentioned somewhere.
	A possible alternative could be in the appendix where the constraints are given in more detail.}

\subsection{Computation of GWP}
\ET{So far the topic of how to compute GWP is not addressed.
	The computation is so basic (values multiplied by volume or weight and aggregated) that is does not really needs explaining.
	However we still need to describe how the final value is reached.
	The value includes GWP for the constituents of the concrete.
	This is then multiplied by the volume of the beam.
	The GWP of the rebars is then added.
	Some points that might need to be discussed:
	1: the general problem of what to include in a relevant analysis of GWP: only the materials? the transport? the whole life cycle?
	2: what is a "fair" GWP value for slack. Slack is a "by-product" of steel manufacturing, so some publications consider it as "trash" so it has zero GWP itself, only what is used as energy on postprocessing.
	This is controversial, as once it is applied in industry it is not only a by-product anymore.}




