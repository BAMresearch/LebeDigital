\subsection{Approximate concrete tensile strength based on compressive strenght}
To reduce the number of required experiments, the tensile strength of concrete is computed using the approximations given in \citeauthor{DIN1992-1-1}.
They are based on the compressive strength, 
\begin{align}
\ft = 
\begin{cases}
0.3 \fc^{\frac{2}{3}}  & \text{for $\fc \le 50$ N/mm$^2$}\\
\ft =2.12 \ln \left( 1 \left( \frac{\fc + 8}{10}\right) \right)& \text{for $\fc > 50$ N/mm$^2$}  \label{eq:tensilstrength}.
\end{cases}
\end{align}
\subsection{$\DOHmax$ based on $\wc$}
...
\subsection{Beam design}

\subsubsection{Maximum bending moment and shear force}
Assuming a simply supported beam with a given length $\beamLength$ in mm, a distributed load $\beamDistrLoad$ in N/mm and a point load $\beamPointLoad$ in N/mm
the maximum bending moment $\beamMaxMoment$ in N/mm$^2$ and the maximum shear force $\beamMaxShearForce$ in N/mm$^2$ are computed as
\begin{align}
	\beamMaxMoment= \beamDistrLoad \frac{\beamLength^2}{8} + \beamPointLoad \frac{\beamLength}{4}\\
	\beamMaxShearForce = \beamDistrLoad \frac{\beamLength}{4} + \frac{\beamPointLoad}{2}
\end{align}
\subsubsection{Computing the minimal required steel reinforcement}
Given a beam with the height $\beamHeight$ in mm, a concrete cover of $\beamCover$ in mm, a steel reinforcement diameter of $\beamSteelDiameter$ in mm for both the longitudinal as well as transversal reinforcement, the effective height in mm is
\begin{align}
	\beamHeightEff = \beamHeight - \beamCover - \frac{3}{2} \beamSteelDiameter
\end{align}
According to the standard safety factors are applied, $\beamTimeSF = 0.85$, $\beamConcreteSF = 1.5$ and $\beamSteelSF = 1$, leading to the design compressive strengths for concrete and steel
\begin{align}
	\beamfcd = \beamTimeSF \frac{\fc }{\beamConcreteSF}\\
	\beamfsd = \frac{\beamfs}{\beamSteelSF}
\end{align}

To be continued...

