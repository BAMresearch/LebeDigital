\subsection{Approximate concrete tensile strength based on compressive strength}
To reduce the number of required experiments, the tensile strength of concrete is computed using the approximations given in \citeauthor{DIN1992-1-1}.
They are based on the compressive strength, 
\begin{align}
\ft = 
\begin{cases}
0.3 \fc^{\frac{2}{3}}  & \text{for $\fc \le 50$ N/mm$^2$}\\
\ft =2.12 \ln \left( 1 \left( \frac{\fc + 8}{10}\right) \right)& \text{for $\fc > 50$ N/mm$^2$}  \label{eq:tensilstrength}.
\end{cases}
\end{align}
\subsection{\texorpdfstring{$\DOHmax$}{Maxiumum DoH} based on \texorpdfstring{$\wc$}{w/c}}
Following \cite{Mills1966fico}, the maximum degree of hydration is given by the function
\begin{align}
	\DOHmax = \dfrac{1.031\,\wc}{0.194 + \wc}.
\end{align}
Contrary to other publications, this function does not assume a fully hydrated cement for a value of $\wc = 0.42$ but assumes that the full potential will usually not be reached.
It is possible that the maximum value might be underestimated.
This should not be a problem, as we are identifying our hydration parameter for a given $\wc$ value.
The relevant aspect is the change relative to the value for which experimental data is given.