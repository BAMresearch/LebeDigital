\subsection{Approximate concrete tensile strength based on compressive strenght}
To reduce the number of required experiments, the tensile strength of concrete is computed using the approximations given in \citeauthor{DIN1992-1-1}.
They are based on the compressive strength, 
\begin{align}
\ft = 
\begin{cases}
0.3 \fc^{\frac{2}{3}}  & \text{for $\fc \le 50$ N/mm$^2$}\\
\ft =2.12 \ln \left( 1 \left( \frac{\fc + 8}{10}\right) \right)& \text{for $\fc > 50$ N/mm$^2$}  \label{eq:tensilstrength}.
\end{cases}
\end{align}
\subsection{$\DOHmax$ based on $\wc$}
Following \cite{Mills1966fico}, the maximum degree of hydration is given by the function
\begin{align}
	\DOHmax = \dfrac{1.031\,\wc}{0.194 + \wc}.
\end{align}
Contrary to other publications, this function does not assume a fully hydrated cement for a value of $\wc = 0.42$ but assumes that the full potential will usually not be reached.
It is possible that the maximum value might be underestimated.
This should not be a problem, as we are identifying our hydration parameter for a given $\wc$ value.
The relevant aspect is the change relative to the value for which experimental data is given.
\subsection{Beam design}
Following design code \citeauthor{DIN1992-1-1} for a singly reinforced beam, meaning a reinforced concrete beam with only reinforcement at the bottom.
The assumed cross section is rectangular

\subsubsection{Maximum bending moment and shear force}
Assuming a simply supported beam with a given length $\beamLength$ in mm, a distributed load $\beamDistrLoad$ in N/mm and a point load $\beamPointLoad$ in N/mm
the maximum bending moment $\beamMaxMoment$ in N/mm$^2$ is computed as
% see Schneider (20. Auflage) 4.2
\begin{align}
	\beamMaxMoment= \beamDistrLoad \frac{\beamLength^2}{8} + \beamPointLoad \frac{\beamLength}{4}
\end{align}
The applied loads are supposed to already incorporate any required safety factors.
\subsubsection{Computing the minimal required steel reinforcement}
Given a beam with the height $\beamHeight$ in mm, a concrete cover of $\beamCover$ in mm, a steel reinforcement diameter of $\beamSteelDiameter$ in mm for both the longitudinal as well as transversal reinforcement, the effective height in mm is
\begin{align}
	\beamHeightEff = \beamHeight - \beamCover - \frac{3}{2} \beamSteelDiameter
\end{align}
According to the German norm standard safety factors are applied, $\beamTimeSF = 0.85$, $\beamConcreteSF = 1.5$ and $\beamSteelSF = 1$ (or 1.15???), leading to the design compressive strength for concrete $\beamfcd$ and  the design tensile yield strength $\beamfsd$ for steel
\begin{align}
	\beamfcd = \beamTimeSF \frac{\fc }{\beamConcreteSF}\\
	\beamfsd = \frac{\beamfs}{\beamSteelSF},
\end{align}
where $\beamfs$ is the steel's tensile yield strength.
% following ravis implementation
% bending moment with stress block - K
To compute the force applied in the compression zone, the need to compute the lever arm of the applied moment $\beamLeverMoment$ with 
\begin{align}
	\beamLeverMoment &= \beamHeightEff(0.5+\sqrt{0.25-0.5 \beamK}),\quad \text{with}\quad
	\beamK = \frac{\beamMaxMoment }{\beamWidth \beamHeightEff^2 \beamfcd}
\end{align}
For singly reinforced beam $K < 0.167$.
Maximum value $\beamLeverMoment = 0.95\beamHeightEff$.
Required steel
\begin{align}
\beamSteelReq = \frac{\beamMaxMoment}{\beamfsd \beamLeverMoment}
\end{align}
Now compare to provided steel area
\begin{align}
	\beamSteelProv = \beamNSteel \pi \frac{\beamSteelDiameter^2}{4} 
\end{align}
Finally 
\begin{align}
	A_{check} = \frac{\beamSteelProv - \beamSteelReq}{\beamSteelProv}
\end{align}
Make pretty and move to appendix
