
\subsection{Notes on Early Age Concrete Model}
Plan is do collect notes, information on the early age concrete model I am implementing.
Currently the plan is to include temperature and humidity and couple them the respective mechanical fields.
I will start with the temperature field.

\subsection{Modeling of the temperature field}
Temperature is generally described as

\begin{align}
	\density \heatCapSpecific \dTdt = \nabla \cdot (\thermCondEff \nabla \temp) + \dQdt \label{eq:heat1}
\end{align}
$\thermCondEff$ is the effective thermal conductivity in Wm$^{-1}$K$^{-1}$.
$C$ is the specific heat capacity.
$\density$ is the density.
$\density \heatCapSpecific$ is the volumetric heat capacity in Jm$^{-3}$K$^{-1}$.
$Q$ is the volumetric heat, due to hydration, it is also called the latent heat of hydration, or the heat source in Jm$^{-3}$.
For now we assume the density, the thermal conductivity and the volumetric heat capacity as constant, however there are models that make them dependent on the temperature, moisture and/or the hydration.


\subsubsection{Degree of hydration\texorpdfstring{ $\DOH$}{}}
The degree of hydration $\DOH$ is defined as the ratio between the cumulative heat $\heat$ at time $\zeit$ and the total theoretical volumetric heat by complete hydration $\heatInf$,
\begin{align}
	\DOH(\zeit) = \frac{\heat(\zeit)}{\heatInf}, \label{eq:doh}
\end{align}
by assuming a linear relation between the degree of hydration and the heat development.
Therefore the time derivative of the heat source $\dot{\heat}$ can be rewritten in terms of $\DOH$, 
\begin{align}
	\dQdt = \dDOHdt \heatInf. \label{eq:qdotalphadot}
\end{align}
There are formulas to approximate total potential heat based on composition, approximated values range between 300 and 600 J/g of binder for different cement types, e.g. Ordinary Portland cement $\heatInf =$ 375–525 or Pozzolanic cement $\heatInf =$ 315–420.  

\subsubsection{Affinity}
The heat release can be modeled based on the chemical affinity $\affinity$ of the binder.
The hydration kinetics can be defined as a function of affinity at a reference temperature $\affinityTemp$ and a temperature dependent scale factor ${\affinityScale}$
\begin{align}
	\dot{\DOH} = \affinityTemp(\DOH){\affinityScale}(\temp)\label{eq:affinitydot}
\end{align}


The reference affinity, based on the degree of hydration is approximated by
\begin{align}
	\affinityTemp(\DOH) = \hydParBone \left(\frac{\hydParBtwo}{\DOHmax} + \DOH\right) (\DOHmax - \DOH)\exp\left(-\hydParEta \frac{\DOH}{\DOHmax}\right)
\end{align}
where $\hydParBone$ and $\hydParBtwo$ are coefficients depending on the binder.
The scale function is given as
\begin{align}
	\affinityScale = \exp\left(-\frac{\activE}{\gasConst}\left(\dfrac{1}{\temp}-\dfrac{1}{\tempRef}\right)\right)
\end{align}
An example function to approximate the maximum degree of hydration based on the water to cement mass ratio $\wc$, by Mills (1966)
\begin{align}
	\DOHmax = \dfrac{1.031\wc}{0.194 + \wc},
\end{align}
this refers to Portland cement. Looking at this function this does not seem to be a good expproximation. Need to find better!!!
This is a simple but probably better option.
\cite{pic_2011_uqso} also includes lot of other wc dependencies...\\
\begin{align}
	\DOHmax &= \frac{\wc}{0.42}\quad{\text{for}}\quad\wc < 0.42\\
	\DOHmax &= 1\quad{\text{for}}\quad\wc \ge 0.42
\end{align}

\subsubsection{Time derivative}
For a start I use a simple backward difference, backward Euler, implicit Euler method and approximate
\begin{align}
	\dot{\temp} =& \dfrac{\tempCurrent-\tempLast}{\Delta \zeit} \quad\text{and}\label{eq:timediscr}\\
	\dot{\DOH} =& \dfrac{\Delta \DOH}{\Delta \zeit}  \quad\text{with}\quad
	\Delta \DOH  = \DOHCurrent- \DOHLast
	\label{eq:timediscr2}
\end{align}
\subsubsection{Formulation}
Using \eqref{eq:qdotalphadot} in \eqref{eq:heat1}
the heat equation is given as
\begin{align}
	\density \heatCapSpecific \frac{\partial \temp}{\partial \zeit} = \nabla \cdot (\thermCondEff \nabla \temp) + \heatInf\frac{\partial  \DOH}{\partial \zeit} 
\end{align}
Now we apply the time discretizations \eqref{eq:timediscr} and \eqref{eq:timediscr2} and drop the index $\currentn$ for readability \eqref{eq:timediscr}
\begin{align}
	\density \heatCapSpecific \temp  - {\Delta \zeit} \nabla \cdot (\thermCondEff \nabla \temp) - \heatInf\Delta \DOH
	= \density \heatCapSpecific \tempLast \label{eq:heat2}
\end{align}
Now, we use \eqref{eq:timediscr2} and \eqref{eq:affinitydot} to get a formulation for $\Delta \DOH$
\begin{align}
	\Delta \DOH = \Delta \zeit \affinityTemp( \DOH)\affinityScale(\temp) \label{eq:deltaalpha}
\end{align}
\subsubsection{Computing \texorpdfstring{$\Delta \DOH$}{change in DoH} at the Gauss-point}
As $\Delta \DOH$ is not a global field, rather locally defined information.
\subsubsection{Solving for \texorpdfstring{$\Delta \DOH$}{change in DoH}}
To solve for $\Delta \DOH$ we define the affinity in terms of $ \DOHLast$ and $\Delta \DOH$
\begin{align}
	\affinityTemp = \hydParBone\exp\left(-\hydParEta \frac{\Delta \DOH+\DOHLast}{\DOHmax}\right) \left(\tfrac{\hydParBtwo}{\DOHmax} + \Delta\DOH+\DOHLast\right) (\DOHmax - \Delta\DOH - \DOHLast).
\end{align}
Now we can solve the nonlinear function 
\begin{align}
	\function(\Delta\DOH) = \Delta\DOH - \Delta \zeit \affinityTemp(\Delta\DOH)\affinityScale(\temp) = 0
\end{align}
using an iterative Newton-Raphson solver. For an effective algorithm we require the tangent of $\function$ with respect to $\Delta\DOH$
\begin{align}
	\dfrac{\partial \function}{\partial \Delta\DOH} = 1 - \Delta \zeit \affinityScale(\temp) \dfrac{\partial\affinityTemp}{\partial \Delta\DOH} \quad\text{with}\\
	\dfrac{\partial\affinityTemp}{\partial \Delta\DOH} = \hydParBone\exp\left(-\hydParEta \frac{\Delta\DOH+\DOHLast}{\DOHmax}\right)\left[
	\DOHmax - \tfrac{\hydParBtwo}{\DOHmax} - 2\Delta\DOH - 2 \DOHLast\right.\quad\quad&\nonumber\\
	+ (\tfrac{\hydParBtwo}{\DOHmax} + \Delta\DOH+\DOHLast)(\Delta\DOH\left.+ \DOHLast - \DOHmax)(\tfrac{\hydParEta}{\DOHmax})\right]&
\end{align}
\subsubsection{Macroscopic tangent}
To incorporate the heat term in the this in the global temperature field, we need to compute to tangent of the term $\heatInf\Delta\DOH$.
Therefore the sensitivity of $\Delta\DOH$ with respect to the temperature $\temp$ needs to be computed $\dfrac{\partial \Delta\DOH}{\partial \temp}$
\begin{align}
	\dfrac{\partial \Delta\DOH}{\partial \temp} = \Delta \zeit \affinityTemp(\DOH)\dfrac{\partial \affinityScale(\temp)}{\partial \temp},\text{ with}\\
	\dfrac{\partial \affinityScale(\temp)}{\partial \temp} = \affinityScale(\temp) \frac{\activE}{\gasConst\temp^2}
\end{align}

\subsection{Coupling Material Properties to Degree of Hydration} \label{appendix:fem_evolution}
\subsubsection{Compressive strength}
The compressive strength in terms of the degree of hydration can be approximated using an exponential function, c.f. \cite{car_2016_mamt},
\begin{align}
	\fc(\DOH) = \DOH(\zeit)^{\strengthCExp} \fcInf \label{eq:fcwrtDOH}
\end{align}
This model has two parameters, $\fcInf$, the compressive strength of the parameter at full hydration, $\DOH = 1$ and $\strengthCExp$ the exponent, which is a purely numerical parameter.

The first parameter could theoretically be obtained through experiments.
However the total hydration can take years, therefore usually only the value after 28 days is obtained.
Therefore instead of somehow estimating $\fcInf$, we can approximate it using the 28 days values of the compressive strength and the degree of hydration

\begin{align}
\fcInf = 	\dfrac{\fcTwentyEight}{{\DOHTwentyEight}^{\strengthCExp}}.
\end{align}
\subsubsection{Young's Modulus}
The publication \cite{car_2016_mamt} proposes a new model for the evolution of the Young's modulus.
Instead of the generalized model \eqref{eq:mechanics-hydration}, the model assumes an initial linear increase of the Young's modulus up to a degree of hydration $\DOHt$,
\begin{align}
	\eMod(\DOH) = 
	\begin{cases}
		\eModInf \frac{\DOH(\zeit)}{\DOHt}{\DOHt}^{\stiffExp}   
		& \text{for $\DOH < \DOHt$}\\
		\eModInf {\DOH(\zeit)}^{\stiffExp}  
		& \text{for $\DOH \ge \DOHt$}
	\end{cases}\label{eq:EwrtDOH}.
\end{align}
Contrary to other publications, no dormant period is assumed.
A challenge with this formulation, is that $\eModInf$ is difficult to obtain.
Standardized testing of the Young's modulus is done after 28 day, $\eModTwentyEight$.
To effectively use these experimental values, $\eModInf$ is approximated as
\begin{align}
	\eModInf = \dfrac{\eModTwentyEight}{{\DOHTwentyEight}^{\stiffExp}}.
\end{align}
To use this formulation, the experimental value of the Young's modulus after 28 days $\eModTwentyEight$ is required, as well as the approximated degree of hydration at that time $\DOHTwentyEight$ which can be approximated using a calibrated hydration simulation.


\subsection{Constraints}
The FEM simulation is used to compute two practical constraints relevant to the precast concrete industry.
At each time step, the worst point is chosen to represent the part, therefore ensuring that the criterion is fulfilled in the whole domain.
The first constraint limits the maximum allowed temperature.
The constraint is computed as the normalized difference between the maximum temperature reached $\tempMax$ and the temperature limit $\tempLimit$ 
\begin{align}
	\FEMConstraintT = \frac{\tempMax - \tempLimit}{\tempLimit}, \label{eq:concstraintT}
\end{align}
where $\FEMConstraintT > 0$ is not admissible, as the temperature limit has been exceeded.\\

The second constraint is the estimated time of demolding.
This is critical, as the manufacturer has a limited number of forms.
The faster the part can be demolded, the faster is can be reused, increasing the output capacity.
On the other hand, the part must not be demolded too early, as it might get damaged while being moved.
To approximate the minimal time of demolding, a constraint is formulated based on the local stresses $\FEMConstraintStress$.
It evaluates the Rankine criterion for the principal tensile stresses, using the yield strength of steel $\beamfs$ and a simplified Drucker-Prager criterion, based on the evolving compressive strength of the concrete $\fc$,
\begin{align}
	\FEMConstraintStress =  \max
	\begin{cases}
		\FEMConstraintRK = \frac{\|\principalStressTension\| - \beamfs}{\beamfs} \\
		\FEMConstraintDP =  \frac{\sqrt{\frac{1}{3} \firstStressInvariant^2 - \secondStressInvariant} - \frac{\fc^3}{\sqrt{3}}}{\fc}
	\end{cases},\label{eq:constraintStress}
\end{align}
where $\FEMConstraintStress > 0$ is not admissible.
In contrast to standard yield surfaces, the value is normalized, to be unit less.
This constraint aims to approximate the compressive failure often simulated with plasticity and the tensile effect of reinforcement steel.
As boundary conditions a simply supported beam under it own weight has been chosen, to approximate possible loading condition while the part is moved.
This constraint is evaluated for each time step in the simulation.
Then the critical point in time is approximated where $\FEMConstraintStress(t_crit) = 0$.
This is now compared and normalized with the ...
% for a multiobjective optimization the time could also be mimizied instead of constrained, using the critical time as KPI
