Precast concrete elements play a critical role in achieving efficient, low cost and sustainable structures.
The controlled production environment allows for higher quality products and enables the mass production of elements.
In the standard design approach, engineers or architects select a structure, estimate the loads, choose mechanical properties, and design the element accordingly. 
If the results are not satisfactory, the required mechanical properties are iteratively adjusted, aiming to improve the design.
This approach is fine, when the choice of mixtures is limited and the expected concrete properties are well known.
There are various published methods to automate this process and optimize the beam design at this level.
Computer aided beam design optimization dates back at least 50 year, e.g. \cite{Haung1967}.
Generally the objective is reducing costs, with the design variables being the beam geometry, the amount and location of the reinforcement and sometimes the compressive strength of the concrete \cite{Chakrabarty_1992, Coello_1997, Pierott_2021, Shobeiri_2023} .
Most publications focus on analytical functions based on norms and well known rules of thumb.
In recent years the use of alternative binders in the concrete mix design has increased, mainly to reduce the environmental impact and cost of concrete but also to improve and modify specific properties.
This is a challenge as the concrete mix is no longer a constant and is itself subjected to optimization.
Known heuristics might no longer apply to the new materials and old design approaches might fail to produce optimal results.
In addition it is not favorable to choose from a predetermined set of possible mixes, as this would either lead to an exaggerated number of required experiments or a limiting subset of the possible design space.
There exist literature studying the optimization of specific concrete properties on the concrete mix \cite{Lisienkova_2021, Kondapally_2022}.
\begin{figure}[b]%
	\centering
	\includegraphics[width=1.0\textwidth]{../figures/\designStandard}
	\caption{Classical design approach, where the required material properties are defined before the mix is defined.}\label{fig:standard_design}
\end{figure}
The objective in that literature is normally to either improve some mechanical property like durability within constraints, or to minimize e.g. the amount of concrete while keeping other properties above a threshold.
When designing elements subjected to various requirements, both on the material and structural level, including workability of the fresh concrete, durability of the structure, maximum acceptable temperature, minimal cost and global warming potential, the optimal solution is not apparent and will change depending on each individual project.
The conventional method of design does not allow for an concurrent optimization of structural measures and concrete mix composition, as the structural  design and the concrete mix design are inversely coupled, c.f. Figure \ref{fig:standard_design}.
The lack of coordination between the designer and the concrete manufacturer can therefore lead to suboptimal solutions, as neither party possesses all the relevant information.
A first step to address these limitations is the incorporation of compressive strength during a optimization in the beam design phase.
Higher compressive strength usually correlates with lager amount of cement and therefore higher cost as well as global warming potential.
This approach has shown promising results in achieving improved structural efficiency while considering environmental impact \cite{dos_Santos_2023}.
To be able to find a part specific optimum, individual data of the manufacturer and specific mix options must be integrated.
Therefore, there is still a need for a comprehensive optimization procedure that can seamlessly integrate concrete mixture optimization and structural simulations, ensuring structurally sound and buildable elements with minimized environmental impact for part specific data.
\\
\begin{figure}[b]%
	\centering
	\includegraphics[width=1.0\textwidth]{../figures/\designProposed}
	\caption{Presented design approach that allows for a holistic optimization.}\label{fig:proposed_workflow}
\end{figure}
%%%%%%%%%%%%%%%%%%%%
In this paper, we present a holistic optimization procedure that combines concrete mixture optimization with the structural response of precast concrete elements, using structural simulations as constraints to ensure structural integrity, limit the maximum temperature and ensure an adequate time of demolding.
This inverts the classical design pipeline.
As a first step the concrete mix is defined.
Based on the output, the beam design is created, c.f. Figure \ref{fig:proposed_workflow}.
The chosen example of this optimization procedure is to reduce the GWP of precast concrete elements. 
By integrating the concrete mixture optimization and structural design processes, engineers can tailor the concrete properties to meet specific requirements of the customer and manufacturer.
This approach opens up possibilities for performance prediction and optimization for new mixture that fall outside the standard range of well-known concrete.
To the best of our knowledge there are no published works that combine the material and structural level in one flexible optimization framework.
In addition to changing the order of the design steps, the proposed framework allows to directly integrate experimental data and propagate the identified uncertainties.
This allows a straight forward integration of new data and and quantification of uncertainties regarding the predictions.
The proposed framework consists of three main parts.
First, an automated and reproducible parameter identification method to calibrate the models.
Second, a gradient-based optimization method for non-differentiable functions, including constraints.
Third, a flexible workflow combining the models and functions required for the respective problem. 
For this publication a well known example of a simply supported, reinforced, rectangular beam  has been chosen.
The design problem was originally published in \cite{everard1966reinforced}.
It has been used to showcase different optimization schemes, e.g. \cite{Chakrabarty_1992}, \cite{Coello_1997}, \cite{Pierott_2021}.
The experimental data used in the parameter identification step is mainly sourced from \cite{gruyaert2011}.
The objective is to reduce the overall global warming potential of the part.
This objective is a particularly meaningful as the cement industry, a major contributor to GWP, accounts for approximately 8\% of the total anthropogenic GWP. 
Reducing the environmental impact of cement production becomes crucial in the pursuit of sustainable construction practices.
In addition, the reduction of cement is also correlated to the reduction of cost, as cement is generally the most expensive component of the concrete mix \cite{Paya_Zaforteza_2009}.
There are three direct ways to reduce the amount of cement.
First, replace the cement with a substitute with a lower carbon footprint.
This can change mechanical properties, but does not necessarily mean a reduction in strength.
Second, increase the amount of aggregates.
This also changes effective properties and needs to be balanced with the workability and the limits due to the applications.
Third, decrease the overall volume of concrete.
In addition, when analyzing the whole life-cycle of concrete, both cost and GWP can be reduced by increasing the durability and therefore extending the lifetime of the object.
To showcase the methods capability, two design variables have been chosen, the height of the beam and the ratio of ordinary Portland cement (OPC) to its replacement binder ground granulated blast furnace slack, a by-product of iron industry.\\
%
The value of this manuscript lies in two main contributions. 
Firstly, it presents the possibility to automatically compute relevant Key Performance Indicators (KPIs) at the structural level, based on input values that incorporate parameters relevant to the concrete mix design, inverting the classical design pipeline.
This allows for a comprehensive evaluation of both structural performance and environmental impact. 
Secondly, the paper details the numerical methods employed to conduct a robust optimization process without derivatives, taking into account uncertainties based on raw experimental data.
\\
% this might change and needs to be adapted
The paper starts  with the theoretical background of the parameter identification method in Section \ref{sec:calibration}.
It is followed by the optimization method in Section \ref{sec:optimization} and some numerical experiments showcasing the method in \ref{sec:numericalexperiments}.
Section \ref{sec:data} gives details on the example problem and an overview of the available experimental data.
The following Section \ref{sec:models} gives an overview of the material models and applied assumption. 
In Section \ref{sec:results} all parts, the experimental data, the calibration method, the numerical models and the optimization framework are combined to demonstrate the effectiveness and practicality of the proposed approach.
The publication finishes with a conclusion and outlook in Section \ref{sec:conclusion}.