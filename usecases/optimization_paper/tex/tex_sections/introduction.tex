General introduction.
Giving background, motivation and state of the art.
I would focus on example with reduction of GWP, as this is a good example where the optimum is difficult to find by local optimization.
Improving GWP usually reduces material properties.
Reduced properties usually increases GWP, as mor ematerial is required, therefore this is an insteresting problem.

The focus of this paper is not on the model applied, these are not newly developed on their own.
The added value of this manuscript is on the one hand showing a method that is able to automatically compute relevant KPI on the structural level, based on input values, including parameters relevant for the concrete mix design.
On the other hand this paper gives details on numerical methods to run a robust (?) optimization method which takes into account uncertainties based on the raw experimental data.

In Figure \ref{fig:workflow} an overview of the workflow is given which is used to compute the KPIs and is the heart of the optimization scheme.
\begin{figure}[h]%
	\centering
	\includegraphics[width=1.0\textwidth]{../figures/\workflowGraph}
	\caption{Testing doit workflow. Figure is created by script. Macro for path is created by other script}\label{fig:workflow}
\end{figure}

Section \ref{sec:models} gives an overview of the material models and approximations used to ... the concrete properties.
This section is not ... as curtting edge but rather a ... 
for each deterministic model applied, there are more advanced and usually more complex .. model can de found in the literature.
The aim here is to show the applicability of the overall method with a sufficient level of sophistication.
More in debth derivation of the models are found in the appendix

Section \ref{sec:calibration} gives insight in the the calibration methods used to identify material parameters and their uncertainties.

Section \ref{sec:optimization} then goes into detail about the optimzation scheme applied in this work.

